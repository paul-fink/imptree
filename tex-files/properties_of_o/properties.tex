\documentclass[a4paper]{article}
\usepackage{amsmath}
% \usepackage[ngerman]{babel}
\usepackage{amssymb}

\begin{document}
In this short paper the properties of the measure for the \emph{optimism} in an interval based entropy comparison are displayed.\\

\section{Definition}

Let the base interval be $[a,b]$ and the one that it is compared to $[c,d]$. The measure for the optimisim is then defined to
\begin{displaymath}
o = \frac{a-c}{d+ |a-c|}
\end{displaymath}
Furthermore to ensure a proper value we define 
\begin{displaymath}
o = 0 :\Leftrightarrow a - c = 0
\end{displaymath}

\section{Properties}


\subsection{Varying c and fixed a and d}
Let us assume that we have the borders $a$ and $d$ fixed and $c$ is varying. There are three different cases for special values of c.\\

The most simple case is for $c = a$,
\begin{displaymath}
o = \frac{a-c}{d+ |a-c|} = \frac{a-a}{d+a-a} = 0 \quad \textrm{.}
\end{displaymath}\\

Another case of interest is when $c \approx d$, i.e. the compared to interval almost collapses to a point. In this case we also need to consider the relation between $a$ and $d$.

Let us first have a look at the absolute value of $o$
\begin{eqnarray*}
|o| & = & \left| \frac{a-d}{d+|a-d|} \right|\\
& =& \frac{|a-d|}{d+|a-d|}\\
& = & \left\{ 
\begin{array}{rl}
\frac{a-d}{a} & \text{if } a > d,\\
\frac{d-a}{2d-a} & \text{if } a \leq d.
\end{array}
\right.\\
& \leq & 1 \quad \forall a,d \quad \textrm{.} 
\end{eqnarray*}

With this in mind we are able to derive bounds for $o$ 
\begin{displaymath}
-1 \leq \frac{a-d}{2d-a} \leq o \leq \frac{a-d}{a} \leq 1
\end{displaymath}
For this view the addition to the definition is necessary to avoid 0-valued denominators.\\


Finally the extreme case when $ c = 0 $ is worth pointing out. As we already covered the point where $c = d$ in the above we assume that $c < d$.
\begin{displaymath}
 o = \frac{a-c}{d+|a-c|} = \frac{a}{d+a}
\end{displaymath}

Again we can derive bounds for $o$ depending on the values of $a$ and $d$.
\begin{displaymath}
0 \leq o < 1.
\end{displaymath}
The lower bound is met for $a=0$ and the upper holds for $a=1$ and $d$ very close to $0$.

\subsection{Varying a and fixed c and d}
There are 3 different case which need to be considered, excluding the trivial of $a \to c$:
\begin{itemize}
\item $a \to 0$
\item $a \to d$
\item $a \to 1$
\end{itemize}
Also the case when the compared to interval collapses to a single point is excluded. However, the results still apply yet the strict inequalities do not longer hold and they change to equalities instead.

For the first case we have
\begin{displaymath}
0 > o \to \frac{-c}{d+c} > -1 \quad \textrm{for } a \to 0\textrm{,}
\end{displaymath}

while the second one is not more complicated as we get
\begin{displaymath}
0 < o \to \frac{d-c}{2d-c} < 1 \quad \textrm{for } a \to d \textrm{,}
\end{displaymath}

and finally for $a \to ent^*$, with $ent^*$ being the entropy of the uniform distribution, we obtain
\begin{displaymath}
\frac{d-c}{2d-c} < o \to \frac{ent^*-c}{d+ent^*-c} < 1 \quad \textrm{for } a \to 1 \textrm{.}
\end{displaymath}

\subsection{Varying d and fixed a and c}
At first we consider the case when $d \to c$, i.e. the interval is going to collapse. It is very similar to the case of $c \to d$ which is reflected in the results:
\begin{displaymath}
-1 \leq \frac{a-c}{2c-a} \leq o \leq \frac{a-c}{a} \leq 1 \quad \textrm{,}
\end{displaymath}
where $o$ is only negative for $a<c$ and positive for $a > c$.\\

Another quite interesting aspect is when $d \to a$ as with $a>d$ the intervals do not overlap,
\begin{displaymath}
0 < o \to \frac{a-c}{2a-c} < 1 \quad \textrm{for } d \to a \textrm{.}
\end{displaymath} 
The last case worth considering is when $d$ is nearing $ent^*$, then the value of $o$ increases as we like to penalize broad intervals

\end{document}
